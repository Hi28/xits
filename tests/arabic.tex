% some useless examples

% register 'rtlm' feature for 'base' mode
\ctxlua{fonts.otf.features.register_base_substitution('rtlm')}

\definefontfeature[virtualmath]      [virtualmath][rtlm=yes]
\definefontfeature[math-text]        [virtualmath][ssty=no]
\definefontfeature[math-script]      [virtualmath][ssty=1,mathsize=yes]
\definefontfeature[math-scriptscript][virtualmath][ssty=2,mathsize=yes]

\starttypescript    [math]     [xits][name]
  \definefontsynonym[MathRoman][file:../xits-math.otf] [features=math\mathsizesuffix]
\stoptypescript
\starttypescript[mainface]
  \definetypeface   [mainface][rm][serif] [termes][default]
  \definetypeface   [mainface][mm][math]  [xits]  [default]
\stoptypescript
\usetypescript[mainface]
\setupbodyfont[mainface]

\Umathcode`٠="7"0"0660
\Umathcode`١="7"0"0661
\Umathcode`٢="7"0"0662
\Umathcode`٣="7"0"0663
\Umathcode`٤="7"0"0664
\Umathcode`٥="7"0"0665
\Umathcode`٦="7"0"0666
\Umathcode`٧="7"0"0667
\Umathcode`٨="7"0"0668
\Umathcode`٩="7"0"0669

\mathdir TRT

\starttext
\startformula
\left(\root{٢} \of{١٥٥}\right)
\stopformula

\startformula
\left[\int^{٥٥}_{\kern-.6em ١٢٣} ٦٦٦^٣\right] % the kern is a temp. hack
\stopformula

\startformula
\left\{\sum^{٥٥}_{١٢٣} ٦٦٦^٣\right\}
\stopformula

\startformula
\sqrt{\sqrt{\sqrt{\sqrt{\sqrt{\sqrt{\sqrt{\sqrt{\sqrt{\sqrt{٥٥}}}}}}}}}}
\stopformula

\startformula
٥ < ٦ > ٤
\stopformula

\startformula
٥ \leq ٦ \geq ٧
\stopformula
\stoptext
